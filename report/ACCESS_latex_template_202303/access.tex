\documentclass{ieeeaccess}
\usepackage{cite}
\usepackage{amsmath,amssymb,amsfonts}
\usepackage{algorithmic}
\usepackage{graphicx}
\usepackage{textcomp}
\def\BibTeX{{\rm B\kern-.05em{\sc i\kern-.025em b}\kern-.08em
    T\kern-.1667em\lower.7ex\hbox{E}\kern-.125emX}}
\begin{document}

\doi{10.1109/ACCESS.2023.0322000}

\title{Preparation of Papers for IEEE ACCESS}
\author{\uppercase{First A. Author}\authorrefmark{1},
\uppercase{Second B. Author}\authorrefmark{2}, and Third C. Author,
Jr.\authorrefmark{3},
\IEEEmembership{Member, IEEE}}

\address[1]{National Institute of Standards and
Technology, Boulder, CO 80305 USA (e-mail: author@boulder.nist.gov)}
\address[2]{Department of Physics, Colorado State University, Fort Collins,
CO 80523 USA (e-mail: author@lamar.colostate.edu)}
\address[3]{Electrical Engineering Department, University of Colorado, Boulder, CO
80309 USA}
\tfootnote{This paragraph of the first footnote will contain support
information, including sponsor and financial support acknowledgment. For
example, ``This work was supported in part by the U.S. Department of
Commerce under Grant BS123456.''}

\markboth
{Author \headeretal: Preparation of Papers for IEEE TRANSACTIONS and JOURNALS}
{Author \headeretal: Preparation of Papers for IEEE TRANSACTIONS and JOURNALS}

\corresp{Corresponding author: First A. Author (e-mail: author@ boulder.nist.gov).}

\begin{abstract}
These instructions give you guidelines for preparing papers for
IEEE Access. Use this document as a template if you are
using \LaTeX. Otherwise, use this document as an
instruction set. The electronic file of your paper will be formatted further
at IEEE. Paper titles should be written in uppercase and lowercase letters,
not all uppercase. Avoid writing long formulas with subscripts in the title;
short formulas that identify the elements are fine (e.g., "Nd--Fe--B"). Do
not write ``(Invited)'' in the title. Full names of authors are preferred in
the author field, but are not required. Put a space between authors'
initials. The abstract must be a concise yet comprehensive reflection of
what is in your article. In particular, the abstract must be self-contained,
without abbreviations, footnotes, or references. It should be a microcosm of
the full article. The abstract must be between 150--250 words. Be sure that
you adhere to these limits; otherwise, you will need to edit your abstract
accordingly. The abstract must be written as one paragraph, and should not
contain displayed mathematical equations or tabular material. The abstract
should include three or four different keywords or phrases, as this will
help readers to find it. It is important to avoid over-repetition of such
phrases as this can result in a page being rejected by search engines.
Ensure that your abstract reads well and is grammatically correct.
\end{abstract}

\begin{keywords}
Enter key words or phrases in alphabetical
order, separated by commas. Autocorrelation, beamforming, communications technology, dictionary learning, feedback, fMRI, mmWave, multipath, system design, multipath, slight fault, underlubrication fault.
\end{keywords}

\titlepgskip=-21pt

\maketitle

\section{Introduction}
\label{sec:introduction}
\PARstart{T}{his} document is a template for \LaTeX. If you are reading a paper or PDF version of this document, please download the LaTeX template or the MS Word
template of your preferred publication from the IEEE Website at \underline
{https://template-selector.ieee.org/secure/templateSelec}\break\underline{tor/publicationType} so you can use it to prepare your manuscript.
If you would prefer to use LaTeX, download IEEE's LaTeX style and sample files
from the same Web page. You can also explore using the Overleaf editor at
\underline
{https://www.overleaf.com/blog/278-how-to-use-overleaf-}\break\underline{with-ieee-collabratec-your-quick-guide-to-getting-started}\break\underline{\#.xsVp6tpPkrKM9}

IEEE will do the final formatting of your paper. If your paper is intended
for a conference, please observe the conference page limits.

\subsection{Related work}
Define abbreviations and acronyms the first time they are used in the text,
even after they have already been defined in the abstract. Abbreviations
such as IEEE, SI, ac, and dc do not have to be defined. Abbreviations that
incorporate periods should not have spaces: write ``C.N.R.S.,'' not ``C. N.
R. S.'' Do not use abbreviations in the title unless they are unavoidable
(for example, ``IEEE'' in the title of this article).

\subsection{Concept}
Use one space after periods and colons. Hyphenate complex modifiers:
``zero-field-cooled magnetization.'' Avoid dangling participles, such as,
``Using \eqref{eq}, the potential was calculated.'' [It is not clear who or what
used \eqref{eq}.] Write instead, ``The potential was calculated by using \eqref{eq},'' or
``Using \eqref{eq}, we calculated the potential.''

Use a zero before decimal points: ``0.25,'' not ``.25.'' Use
``cm$^{3}$,'' not ``cc.'' Indicate sample dimensions as ``0.1 cm
$\times $ 0.2 cm,'' not ``0.1 $\times $ 0.2 cm$^{2}$.'' The
abbreviation for ``seconds'' is ``s,'' not ``sec.'' Use
``Wb/m$^{2}$'' or ``webers per square meter,'' not
``webers/m$^{2}$.'' When expressing a range of values, write ``7 to
9'' or ``7--9,'' not ``7$\sim $9.''

A parenthetical statement at the end of a sentence is punctuated outside of
the closing parenthesis (like this). (A parenthetical sentence is punctuated
within the parentheses.) In American English, periods and commas are within
quotation marks, like ``this period.'' Other punctuation is ``outside''!
Avoid contractions; for example, write ``do not'' instead of ``don't.'' The
serial comma is preferred: ``A, B, and C'' instead of ``A, B and C.''

If you wish, you may write in the first person singular or plural and use
the active voice (``I observed that $\ldots$'' or ``We observed that $\ldots$''
instead of ``It was observed that $\ldots$''). Remember to check spelling. If
your native language is not English, please get a native English-speaking
colleague to carefully proofread your paper.

Try not to use too many typefaces in the same article. Also please remember that MathJax
can't handle really weird typefaces.

\subsection{Your own scheduling algorithm}
Number equations consecutively with equation numbers in parentheses flush
with the right margin, as in \eqref{eq}. To make your equations more
compact, you may use the solidus (~/~), the exp function, or appropriate
exponents. Use parentheses to avoid ambiguities in denominators. Punctuate
equations when they are part of a sentence, as in
\begin{equation}E=mc^2.\label{eq}\end{equation}

The following 2 equations are used to test
your LaTeX compiler's math output. Equation (2) is your LaTeX compiler' output. Equation (3) is an image of what (2) should look like.
Please make sure that your equation (2) matches (3) in terms of symbols and characters' font style (Ex: italic/roman).

\begin{align*} \frac{47i+89jk\times 10rym \pm 2npz }{(6XYZ\pi Ku) Aoq \sum _{i=1}^{r} Q(t)} {\int\limits_0^\infty \! f(g)\mathrm{d}x}  \sqrt[3]{\frac{abcdelqh^2}{ (svw) \cos^3\theta }} . \tag{2}\end{align*}

$\hskip-7pt$\includegraphics[scale=0.52]{equation3.png}

Be sure that the symbols in your equation have been defined before the
equation appears or immediately following. Italicize symbols ($T$ might refer
to temperature, but T is the unit tesla). Refer to ``\eqref{eq},'' not ``Eq. \eqref{eq}''
or ``equation \eqref{eq},'' except at the beginning of a sentence: ``Equation \eqref{eq}
is $\ldots$ .''

Please use ``soft'' (e.g., \verb|\eqref{Eq}|) cross references instead
of ``hard'' references (e.g., \verb|(1)|). That will make it possible
to combine sections, add equations, or change the order of figures or
citations without having to go through the file line by line.

Please don't use the \verb|{eqnarray}| equation environment. Use
\verb|{align}| or \verb|{IEEEeqnarray}| instead. The \verb|{eqnarray}|
environment leaves unsightly spaces around relation symbols.

Please note that the \verb|{subequations}| environment in {\LaTeX}
will increment the main equation counter even when there are no
equation numbers displayed. If you forget that, you might write an
article in which the equation numbers skip from (17) to (20), causing
the copy editors to wonder if you've discovered a new method of
counting.

{\BibTeX} does not work by magic. It doesn't get the bibliographic
data from thin air but from .bib files. If you use {\BibTeX} to produce a
bibliography you must send the .bib files.

{\LaTeX} can't read your mind. If you assign the same label to a
subsubsection and a table, you might find that Table I has been cross
referenced as Table IV-B3.

{\LaTeX} does not have precognitive abilities. If you put a
\verb|\label| command before the command that updates the counter it's
supposed to be using, the label will pick up the last counter to be
cross referenced instead. In particular, a \verb|\label| command
should not go before the caption of a figure or a table.

Do not use \verb|\nonumber| inside the \verb|{array}| environment. It
will not stop equation numbers inside \verb|{array}| (there won't be
any anyway) and it might stop a wanted equation number in the
surrounding equation.

\section{Methodology}

\subsection{Software structure of your system}

Use either SI (MKS) or CGS as primary units. (SI units are strongly
encouraged.) English units may be used as secondary units (in parentheses).
This applies to papers in data storage. For example, write ``15
Gb/cm$^{2}$ (100 Gb/in$^{2})$.'' An exception is when
English units are used as identifiers in trade, such as ``3$^{1\!/\!2}$-in
disk drive.'' Avoid combining SI and CGS units, such as current in amperes
and magnetic field in oersteds. This often leads to confusion because
equations do not balance dimensionally. If you must use mixed units, clearly
state the units for each quantity in an equation.

The SI unit for magnetic field strength $H$ is A/m. However, if you wish to use
units of T, either refer to magnetic flux density $B$ or magnetic field
strength symbolized as $\mu _{0}H$. Use the center dot to separate
compound units, e.g., ``A$\cdot $m$^{2}$.''


\Figure[t!](topskip=0pt, botskip=0pt, midskip=0pt){fig1.png}
{ \textbf{Magnetization as a function of applied field.
It is good practice to explain the significance of the figure in the caption.}\label{fig1}}

\subsection{Testing cases/Assumptions}
The word ``data'' is plural, not singular. The subscript for the
permeability of vacuum $\mu _{0}$ is zero, not a lowercase letter
``o.'' The term for residual magnetization is ``remanence''; the adjective
is ``remanent''; do not write ``remnance'' or ``remnant.'' Use the word
``micrometer'' instead of ``micron.'' A graph within a graph is an
``inset,'' not an ``insert.'' The word ``alternatively'' is preferred to the
word ``alternately'' (unless you really mean something that alternates). Use
the word ``whereas'' instead of ``while'' (unless you are referring to
simultaneous events). Do not use the word ``essentially'' to mean
``approximately'' or ``effectively.'' Do not use the word ``issue'' as a
euphemism for ``problem.'' When compositions are not specified, separate
chemical symbols by en-dashes; for example, ``NiMn'' indicates the
intermetallic compound Ni$_{0.5}$Mn$_{0.5}$ whereas
``Ni--Mn'' indicates an alloy of some composition
Ni$_{x}$Mn$_{1-x}$.

Be aware of the different meanings of the homophones ``affect'' (usually a
verb) and ``effect'' (usually a noun), ``complement'' and ``compliment,''
``discreet'' and ``discrete,'' ``principal'' (e.g., ``principal
investigator'') and ``principle'' (e.g., ``principle of measurement''). Do
not confuse ``imply'' and ``infer.''

Prefixes such as ``non,'' ``sub,'' ``micro,'' ``multi,'' and ``ultra'' are
not independent words; they should be joined to the words they modify,
usually without a hyphen. There is no period after the ``et'' in the Latin
abbreviation ``\emph{et al.}'' (it is also italicized). The abbreviation ``i.e.,'' means
``that is,'' and the abbreviation ``e.g.,'' means ``for example'' (these
abbreviations are not italicized).

A general IEEE styleguide is available at \break
\underline{http://www.ieee.org/authortools}.

\subsection{Performance analysis}
what what what


\section{Results/graphs/figures discussion}
\label{sec:guidelines}

The following list outlines the different types of graphics published in
IEEE journals. They are categorized based on their construction, and use of
color/shades of gray:

{Figures that are meant to appear in color, or shades of black/gray. Such
figures may include photographs, illustrations, multicolor graphs, and
flowcharts. For multicolor graphs, please avoid any gray backgrounds or shading, as well as screenshots, instead export the graph from the program used to collect the data.}

{Figures that are composed of only black lines and shapes. These figures
should have no shades or half-tones of gray, only black and white.}

{Author photographs should be included with the author biographies located at the end of the article underneath References. }

{Data charts which are typically black and white, but sometimes include
color.}

\begin{table}
\caption{\textbf{Units for Magnetic Properties}}
\label{table}
\setlength{\tabcolsep}{3pt}
\begin{tabular}{|p{25pt}|p{75pt}|p{115pt}|}
\hline
Symbol&
Quantity&
Conversion from Gaussian and \par CGS EMU to SI $^{\mathrm{a}}$ \\
\hline
$\Phi $&
magnetic flux&
1 Mx $\to  10^{-8}$ Wb $= 10^{-8}$ V$\cdot $s \\
$B$&
magnetic flux density, \par magnetic induction&
1 G $\to  10^{-4}$ T $= 10^{-4}$ Wb/m$^{2}$ \\
$H$&
magnetic field strength&
1 Oe $\to  10^{3}/(4\pi )$ A/m \\
$m$&
magnetic moment&
1 erg/G $=$ 1 emu \par $\to 10^{-3}$ A$\cdot $m$^{2} = 10^{-3}$ J/T \\
$M$&
magnetization&
1 erg/(G$\cdot $cm$^{3}) =$ 1 emu/cm$^{3}$ \par $\to 10^{3}$ A/m \\
4$\pi M$&
magnetization&
1 G $\to  10^{3}/(4\pi )$ A/m \\
$\sigma $&
specific magnetization&
1 erg/(G$\cdot $g) $=$ 1 emu/g $\to $ 1 A$\cdot $m$^{2}$/kg \\
$j$&
magnetic dipole \par moment&
1 erg/G $=$ 1 emu \par $\to 4\pi \times  10^{-10}$ Wb$\cdot $m \\
$J$&
magnetic polarization&
1 erg/(G$\cdot $cm$^{3}) =$ 1 emu/cm$^{3}$ \par $\to 4\pi \times  10^{-4}$ T \\
$\chi , \kappa $&
susceptibility&
1 $\to  4\pi $ \\
$\chi_{\rho }$&
mass susceptibility&
1 cm$^{3}$/g $\to  4\pi \times  10^{-3}$ m$^{3}$/kg \\
$\mu $&
permeability&
1 $\to  4\pi \times  10^{-7}$ H/m \par $= 4\pi \times  10^{-7}$ Wb/(A$\cdot $m) \\
$\mu_{r}$&
relative permeability&
$\mu \to \mu_{r}$ \\
$w, W$&
energy density&
1 erg/cm$^{3} \to  10^{-1}$ J/m$^{3}$ \\
$N, D$&
demagnetizing factor&
1 $\to  1/(4\pi )$ \\
\hline
\multicolumn{3}{p{251pt}}{Vertical lines are optional in tables. Statements that serve as captions for
the entire table do not need footnote letters. }\\
\multicolumn{3}{p{251pt}}{$^{\mathrm{a}}$Gaussian units are the same as cg emu for magnetostatics; Mx
$=$ maxwell, G $=$ gauss, Oe $=$ oersted; Wb $=$ weber, V $=$ volt, s $=$
second, T $=$ tesla, m $=$ meter, A $=$ ampere, J $=$ joule, kg $=$
kilogram, H $=$ henry.}
\end{tabular}
\label{tab1}
\end{table}

Figures compiled of more than one sub-figure presented side-by-side, or
stacked. If a multipart figure is made up of multiple figure
types (one part is lineart, and another is grayscale or color) the figure
should meet the stricter guidelines.

subsection{File Formats For Graphics}\label{formats}
Format and save your graphics using a suitable graphics processing program
that will allow you to create the images as PostScript (.PS), Encapsulated
PostScript (.EPS), Tagged Image File Format (.TIFF), Portable Document
Format (.PDF), Portable Network Graphics (.PNG), or Metapost (.MPS), sizes them, and adjusts
the resolution settings. When
submitting your final paper, your graphics should all be submitted
individually in one of these formats along with the manuscript.

subsection{Sizing of Graphics}
Most charts, graphs, and tables are one column wide (3.5 inches/88
millimeters/21 picas) or page wide (7.16 inches/181 millimeters/43
picas). The maximum depth a graphic can be is 8.5 inches (216 millimeters/54
picas). When choosing the depth of a graphic, please allow space for a
caption. Figures can be sized between column and page widths if the author
chooses, however it is recommended that figures are not sized less than
column width unless when necessary.

There is currently one publication with column measurements that do not
coincide with those listed above. Proceedings of the IEEE has a column
measurement of 3.25 inches (82.5 millimeters/19.5 picas).

The final printed size of author photographs is exactly
1 inch wide by 1.25 inches tall (25.4 millimeters$\,\times\,$31.75 millimeters/6
picas$\,\times\,$7.5 picas). Author photos printed in editorials measure 1.59 inches
wide by 2 inches tall (40 millimeters$\,\times\,$50 millimeters/9.5 picas$\,\times\,$12
picas).

subsection{Resolution }
The proper resolution of your figures will depend on the type of figure it
is as defined in the ``Types of Figures'' section. Author photographs,
color, and grayscale figures should be at least 300dpi. Line art, including
tables should be a minimum of 600dpi.

subsection{Vector Art}
In order to preserve the figures' integrity across multiple computer
platforms, we accept files in the following formats: .EPS/.PDF/.PS. All
fonts must be embedded or text converted to outlines in order to achieve the
best-quality results.

subsection{Color Space}
The term color space refers to the entire sum of colors that can be
represented within the said medium. For our purposes, the three main color
spaces are Grayscale, RGB (red/green/blue) and CMYK
(cyan/magenta/yellow/black). RGB is generally used with on-screen graphics,
whereas CMYK is used for printing purposes.

All color figures should be generated in RGB or CMYK color space. Grayscale
images should be submitted in Grayscale color space. Line art may be
provided in grayscale OR bitmap colorspace. Note that ``bitmap colorspace''
and ``bitmap file format'' are not the same thing. When bitmap color space
is selected, .TIF/.TIFF/.PNG are the recommended file formats.

subsection{Accepted Fonts Within Figures}
When preparing your graphics IEEE suggests that you use of one of the
following Open Type fonts: Times New Roman, Helvetica, Arial, Cambria, and
Symbol. If you are supplying EPS, PS, or PDF files all fonts must be
embedded. Some fonts may only be native to your operating system; without
the fonts embedded, parts of the graphic may be distorted or missing.

A safe option when finalizing your figures is to strip out the fonts before
you save the files, creating ``outline'' type. This converts fonts to
artwork what will appear uniformly on any screen.

subsection{Using Labels Within Figures}

subsubsection{Figure Axis labels }
Figure axis labels are often a source of confusion. Use words rather than
symbols. As an example, write the quantity ``Magnetization,'' or
``Magnetization M,'' not just ``M.'' Put units in parentheses. Do not label
axes only with units. As in Fig. 1, for example, write ``Magnetization
(A/m)'' or ``Magnetization (A$\cdot$m$^{-1}$),'' not just ``A/m.'' Do not label axes with a ratio of quantities and
units. For example, write ``Temperature (K),'' not ``Temperature/K.''

Multipliers can be especially confusing. Write ``Magnetization (kA/m)'' or
``Magnetization (10$^{3}$ A/m).'' Do not write ``Magnetization
(A/m)$\,\times\,$1000'' because the reader would not know whether the top
axis label in Fig. 1 meant 16000 A/m or 0.016 A/m. Figure labels should be
legible, approximately 8 to 10 point type.

subsubsection{Subfigure Labels in Multipart Figures and Tables}
Multipart figures should be combined and labeled before final submission.
Labels should appear centered below each subfigure in 8 point Times New
Roman font in the format of (a) (b) (c).

subsection{File Naming}
Figures (line artwork or photographs) should be named starting with the
first 5 letters of the author's last name. The next characters in the
filename should be the number that represents the sequential
location of this image in your article. For example, in author
``Anderson's'' paper, the first three figures would be named ander1.tif,
ander2.tif, and ander3.ps.

Tables should contain only the body of the table (not the caption) and
should be named similarly to figures, except that `.t' is inserted
in-between the author's name and the table number. For example, author
Anderson's first three tables would be named ander.t1.tif, ander.t2.ps,
ander.t3.eps.

Author photographs should be named using the first five characters of the
pictured author's last name. For example, four author photographs for a
paper may be named: oppen.ps, moshc.tif, chen.eps, and duran.pdf.

If two authors or more have the same last name, their first initial(s) can
be substituted for the fifth, fourth, third$\ldots$ letters of their surname
until the degree where there is differentiation. For example, two authors
Michael and Monica Oppenheimer's photos would be named oppmi.tif, and
oppmo.eps.

subsection{Referencing a Figure or Table Within Your Paper}
When referencing your figures and tables within your paper, use the
abbreviation ``Fig.'' even at the beginning of a sentence. Figures should be numbered with Arabic Numerals.
Do not abbreviate ``Table.'' Tables should be numbered with Roman Numerals.

subsection{Submitting Your Graphics}
Because IEEE will do the final formatting of your paper,
you do not need to position figures and tables at the top and bottom of each
column. In fact, all figures, figure captions, and tables can be placed at
the end of your paper. In addition to, or even in lieu of submitting figures
within your final manuscript, figures should be submitted individually,
separate from the manuscript in one of the file formats listed above in
Section \ref{formats}. Place figure captions below the figures; place table titles
above the tables. Please do not include captions as part of the figures, or
put them in ``text boxes'' linked to the figures. Also, do not place borders
around the outside of your figures.

subsection{Color Processing/Printing in IEEE Journals}
All IEEE Transactions, Journals, and Letters allow an author to publish
color figures on IEEE {\it Xplore}$\circledR$\ at no charge, and automatically
convert them to grayscale for print versions. In most journals, figures and
tables may alternatively be printed in color if an author chooses to do so.
Please note that this service comes at an extra expense to the author. If
you intend to have print color graphics, include a note with your final
paper indicating which figures or tables you would like to be handled that
way, and stating that you are willing to pay the additional fee.

\section{Conclusion}
Although a conclusion may review the  main points of the paper, do not replicate the abstract as the conclusion. A
conclusion might elaborate on the importance of the work or suggest
applications and extensions.

If you have multiple appendices, use the $\backslash$appendices command below. If you have only one appendix, use
$\backslash$appendix[Appendix Title]

\appendices
\section{\break source code file strcuture}
Number footnotes separately in superscript numbers.\footnote{It is recommended that footnotes be avoided (except for
the unnumbered footnote with the receipt date on the first page). Instead,
try to integrate the footnote information into the text.} Place the actual
footnote at the bottom of the column in which it is cited; do not put
footnotes in the reference list (endnotes). Use letters for table footnotes
(see Table \ref{table}).

\section{\break sample outputs}
The preferred spelling of the word ``acknowledgment'' in American English is
without an ``e'' after the ``g.'' Use the singular heading even if you have
many acknowledgments. Avoid expressions such as ``One of us (S.B.A.) would
like to thank $\ldots$ .'' Instead, write ``F. A. Author thanks $\ldots$ .'' In most
cases, sponsor and financial support acknowledgments are placed in the
unnumbered footnote on the first page, not here.


\begin{thebibliography}{00}

\bibitem{b1} G. O. Young, ``Synthetic structure of industrial plastics,'' in \emph{Plastics,} 2\textsuperscript{nd} ed., vol. 3, J. Peters, Ed. New York, NY, USA: McGraw-Hill, 1964, pp. 15--64.

\end{thebibliography}

\EOD

\end{document}

%%% Local Variables:
%%% mode: LaTeX
%%% TeX-master: t
%%% End:
